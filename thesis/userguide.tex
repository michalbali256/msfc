\chapter{User Guide}
\label{userguide}
In this appendix, we provide a user guide for building the ns-3 with the implementation of MSFC as well as documentation of the benchmark we used to evaluate it.

The modified ns-3 source codes are available in Attachment A

\section{Build}

Ns-3 is supported on Linux. The only prerequisites for building the ns-3 are a c++ compiler --- g++ or clang and python interpreter.

We can build the ns-3 using following commands:

\begin{lstlisting}[language=bash,caption={}]
cd msfc/ns-allinone-3.28/ns-3.28/
./waf -d optimized configure
./waf
\end{lstlisting}

Further detailed instructions are available on the official ns-3 website\footnote{\url{https://www.nsnam.org/~pdbarnes/docs-related/build.html}}.

\section{Running simulation}

To run our benchmark that we used to evaluate MSFC, we need to execute following commands:

\begin{lstlisting}[language=bash,caption={}]
cd msfc/ns-allinone-3.28/ns-3.28/
./waf [--cwd=outputDirectory] --run "msfc-benchmark [arguments]"
\end{lstlisting}

The \texttt{outputDirectory} is the directory where the results of the simulation will be available once the it is ended.

There are two places where the benchmark is configured: a file with description of flow types and command line arguments. The description of flow types is mandatory. Each line of the file defines one flow type. There are 9 entries separated by spaces on each line in the following order:

\begin{enumerate}
	\item Name
	\item Transport protocol (TCP/UDP)
	\item Count ratio
	\item Data rate
	\item Priority
	\item Whether it is only download or both download and upload (bi/one)
	\item Size of one packet of the flow in bytes
	\item an identifier of ns-3 random variable class that determines the off time of application
	\item an identifier of ns-3 random variable class that determines the off time of application
\end{enumerate}

The Name is just identifier of the flow type that will be displayed in the results. For the description of the rest of the flow type characteristics excapt random variables, refer to the simulation testbed in \autoref{testbed}.

The random variable characteristics allows any random variable that is implemented in ns-3. We only used \\ \texttt{\small ns3::ConstantRandomVariable[Constant=<number>]}, which always returns \\ \texttt{\small number} and \texttt{ ns3::NormalRandomVariable[Mean=1|Variance=1|Bound=1]}, which only returns random numbers between 0 and 2, with normal distribution with mean 1 and variance 1. For the full list of ns-3 random variables with various distributions see the ns-3 manual\footnote{\url{https://www.nsnam.org/docs/release/3.28/manual/ns-3-manual.pdf}}.

The Some parameters can be changed using command line parameters. All the arguments and their values are case-sensitive. Here is the list of them: 

\begin{itemize}
	\item queueDiscType --- the traffic scheduler that will be installed to all nodes. Possible values are PfifoFast, CoDel, FqCoDel, Msfc.
	\item simDuration --- the duration of the simulation in seconds.
	\item connectionDatarate --- sets the bandwidth of the point-to-point links of the ISP tree.
	\item connectionDelay --- sets the delay of the point-to-point links of the ISP tree"
	\item serversDatarate --- sets the bandwidth of the point-to-point link from the tree to the server.
	\item serversDelay --- sets the delay between the server and the root of the ISP tree.
	\item randomPriority --- when set to true, the priorities of individual flows (not the flow types) is determined randomly.
	\item numberOfPrios --- if randomPriority is set to true, this integer determines the number of priorities used.
	\item flowInFileName --- the location of the flow types configuration file. Defaults to \texttt{\small ../flow\_types.in}, so it is recommended to set the cwd argument to a folder in the ns-3.28 folder.
	\item appCount --- the total number of applications generating traffic installed in the whole simulation (two applications at the opposite side of a bidirectional flow count as one).
	\item msfcMultiplier --- the \textsc{Ratio} parameter of MSFC. Has no effect if queueDiscType is not set to 'Msfc'.
\end{itemize}

An example command for running a simulation:

\vspace{3mm}
\texttt{\small
./waf --results --run "msfc-benchmark --simDuration=100 \\ --connectionDatarate=100Mbps --serversDatarate=1000Mbps \\ --randomPriority=0 --appCount=280 --queueDiscType=FqCoDel"
}
\vspace{3mm}

The command starts simulation evaluating FQ CoDel, that runs for 100 (simulated) seconds, it sets the bandwiths of the point-to-point links to 100 Mbps and 1000 Mbps. The flow type characteristics will be loaded from location \texttt{\small ../flow\_types.in}. There will be total of 280 flows in the down direction (2 applications per client node). 

\section{Results}

The benchmark generates several files with simulation results into the working directory, that is set by the \texttt{cwd} argument of \texttt{waf}. All the files are prefixed with the name of queueing discipline that was evaluated, so it is safe to run simulations with different queueDiscTypes in the same working directory. There are XX{X} types of files:

\begin{itemize}
	\item \textless queueDiscType\textgreater -apps-assign.txt --- Contains list of all installed applications with their priorities and the assigned clients. It may be useful, when simulation is run with random priorities. 
	\item \textless queueDiscType\textgreater -flowMonitor.xml --- Contains XML output of FlowMonitor, which encodes detailed information about all flows in the simulation in histograms.
	\item \textless queueDiscType\textgreater -\textless direction\textgreater.all --- Contains overall results of the simulation. There is average delay, average jitter, number of lost packets and average throughput. Additionally, there are averages by priorities and by types of flows.
	\item \textless queueDiscType\textgreater -total-goodput-\textless direction\textgreater.txt --- Contains dependency of total goodput of the simulation on time in the \emph{direction}.
	\item \textless queueDiscType\textgreater -overall-delay-\textless direction\textgreater.det --- Contains information about delay of all packets in the \emph{direction}.
	\item \textless queueDiscType\textgreater -overall-jitter-\textless direction\textgreater.det --- Contains information about jitter of all packets in the \emph{direction}.
	\item \textless queueDiscType\textgreater -delay-prio\textless P\textgreater -\textless direction\textgreater.det --- Contains information about delay of packets from flows with priority \emph{P} in the \emph{direction}.
	\item \textless queueDiscType\textgreater -jitter-prio\textless P\textgreater -\textless direction\textgreater.det --- Contains information about jitter of packets from flows with priority \emph{P} in the \emph{direction}.
	\item \textless queueDiscType\textgreater -type\textless T\textgreater -delay-\textless direction\textgreater.det --- Contains information about delay of packets from flows of type \emph{T} in the \emph{direction}.
	\item \textless queueDiscType\textgreater -type\textless T\textgreater -jitter-\textless direction\textgreater.det --- Contains information about jitter of packets from flows of type \emph{T} in the \emph{direction}.
\end{itemize}

\section{Results analysis}

The .det files contain raw XML data organised in histograms and the data from flows are simply concatenated. However, we provide a simple bash/python script, that can be used to analyse the .det files and generate violin graphs seen in the thesis. The script in located in \texttt{\small msfc/ns-allinone-3.28/ns-3.28/ \\ shs/plot.sh}. It takes 3 parameters

\vspace{3mm}
\texttt{\small
plot.sh <suffix> <upperLimit> <detail>
}
\vspace{3mm}

\texttt{plot.sh} processes all .det files with suffix \emph{suffix}. Then it generates violin plots using python matplotlib library using the detail provided. Additionally, all values from .det files, that are higher than \emph{upperLimit} are ommited during the plot generation.










































