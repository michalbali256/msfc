\chapter*{Introduction}
\addcontentsline{toc}{chapter}{Introduction}

A traffic scheduler determines when, if and in what order packets leave device. Although a traffic scheduler is present in every device connected to the Internet, it is more important in routers, which handle much more traffic. It has critical impact on performance of packet-switched networks, which naturally makes it an object of interest of Internet Service Providers (ISP).  Not only do they need good physical infrastructure, they also need traffic scheduling to maximize network utilisation and ensure fairness among all clients and overall quality of service. Especially the QoS is provided by differentiating services, since individual services have different requirements for network resources.

ISPs often use traffic schedulers like HTB or HFSC in this situation. They allocate bandwidth to services exactly as the user needs. However, their robustness predetermines them for usage at the edges of networks, because it is impossible for the network administrators to figure out the right configuration for each node. Unfortunately, that is not a desired characteristics, since such schedulers do not have information about the inner nodes and thus cannot react to network deficiencies.

In this thesis, we design a traffic scheduler simple enough to be placed at the exact bottlenecks of networks where it can precisely react to network problems and prevent impact on the quality of delivered service. Priority class is only represented by one number, opposed to hierarchical structure or service curves that apply to HTB and HFSC. The only configuration an user has to provide is a packet classifier. We build on previous research in the area --- the design combines ideas of Controlled Delay (CoDel) and Stochastic Fair Queueing (SFQ) and uses Deficit Round-robin to prioritize more important traffic classes. 


\section*{Related Work}


\section*{Layout of this Thesis}


