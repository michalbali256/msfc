\chapter*{Introduction}
\addcontentsline{toc}{chapter}{Introduction}

A traffic scheduler determines when, if and in what order packets leave device. Although a traffic scheduler is present in every device connected to the Internet, it is more important in routers, which handle much more traffic. It has critical impact on performance of packet-switched networks, which naturally makes it an object of interest of Internet Service Providers (ISP).  Not only do they need good physical infrastructure, they also need traffic scheduling to maximize network utilisation and ensure fairness among all clients and overall quality of service. Especially the QoS is provided by differentiating services, since individual services have different requirements for network resources.

ISPs often use traffic schedulers like HTB or HFSC in this situation. They allow  bandwidth allocation to services exactly as the user specifies. However, their robustness predetermines them for usage at the edges of networks, because it is impossible for the network administrators to figure out the right configuration for each node. Unfortunately, that is not a desired characteristics, since such schedulers do not have information about the inner nodes and thus cannot react to network deficiencies.

In this thesis, we design a traffic scheduler simple enough to be placed at the exact bottlenecks of networks where it can precisely react to network problems and prevent impact on the quality of delivered service. Priority class is only represented by one number, opposed to hierarchical structure or service curves that apply to HTB and HFSC. The only configuration an user has to provide is a packet classifier. We build on previous research in the area --- the design combines ideas of Controlled Delay (CoDel) and Stochastic Fair Queueing (SFQ) and uses Deficit Round-robin to prioritize more important traffic classes. 


\section*{Related Work}
Researchers have realised the potential of traffic scheduling a long time ago and many traffic schedulers have emerged addressing different issues. The bufferbloat was first addressed by Random Early Detection (RED) \cite{Floyd:1993:RED:169931.169935}, that used average queue occupancy as marker of full queue. Its later improvements include Advanced RED \cite{Floyd01adaptivered:}, Robust RED \cite{RRED}, and CHoKe \cite{pan2000choke}. In 2012, Nichols, Kathleen and Van Jacobson proposed Controlled Delay (CoDel) algorithm, that uses time packets spend in queue as indication of bufferbloat.

Another issue traffic schedulers address is fair approach to all users of the network. The concept of using one queue per flow is called Fair Queueing (FQ) and Nagle first proposed in \cite{Nagle:FQ}. Stochastic FQ (SFQ) \cite{SFQ} uses hash to distinct flows from each other. Bit-by-bit round-robin \cite{demers1989analysis} ensured absolute fairness for all flows and Deficit round-robin (DRR) \cite{EffDRR} approximates it to achieve O(1) algorithm. FQ CoDel \cite{fq_codel} combines CoDel and DRR.

The last category of traffic schedulers allows user to precisely  define the shape of the traffic. This includes and Class Based Queueing (CBQ) \cite{CBQ}, Hierarchical Token Bucket (HTB) \cite{HTB},  Hierarchical Fair Service Curve (HFSC) \cite{HFSC}.

\section*{Layout of this Thesis}
This thesis is structered as follows. First chapter offers an overview of previous research in traffic scheduling. It presents issues of the area as well as principles and algorithms that address them. Second chapter describes the design of the scheduler we study in this thesis and describes its Linux and ns-3 implementations. Third chapter is dedicated to simulations we performed in order to evaluate the scheduler performance and compare it to other traffic schedulers.

