\chapter*{Introduction}
\addcontentsline{toc}{chapter}{Introduction}

A \todo{network} traffic scheduler determines when, if and in what order \todo{the} packets \todo{what kind of packets? asi chces: the packets in packet-switching networks} packets leave \todo{what kind of device? network connection device} device. Although a traffic scheduler is present in every device connected to the Internet \todo{asi chces rict: some kind of traffic scheduler is present in almost every device...}, it is more important in routers \todo{dulezitej je uplne stejne vsude (jinak bude 99\% drop rate! :D), ale v routerech ma komplikovanejsi ukol}, which handle much more traffic \todo{chces rict higher bitrates; zaroven bych tady nakous ze celej i ruznejm potizim ktery ohrozujou QoS}.

It has \todo{Since the traffic scheduling has...?} critical impact on the performance of packet-switched networks, which naturally makes it \todo{...it naturally becomes...} an object of interest of Internet Service Providers (ISPs): \todo{odsud sem to proste prepsal jinejma slovama tak jak si myslim ze to chces rict} Correct scheduling of packets in traffic flows allows them to maximize the utilisation of their available network resources, ensure the required fairness among their clients, prevent impact of misbehaving clients of malfunctioning hardware, and improve overall quality of service.

The term \emph{quality of service} (often shortened as QoS) is commonly used to denote an informal compound measure of various aspects of packet transfer service, including the available bandwidth, latency, packet loss, and others. Service differentiation, the ability of the QoS-providing mechanisms to satisfy different (possibly incompatible) requirements of different network traffic types, is an important part of the QoS.

ISPs often use classful hierarchical traffic schedulers like HTB or HFSC to achieve the differentiation of service quality. These allow precise bandwidth allocation to any recognizable type of service. However, their design predetermines them for centralized usage at the backbone of the network, which implies two main deficiencies:
\begin{enumerate}
\item The network administrator needs to know exact properties of the network to be able to tune it to optimal performance, which is impossible if the network properties change often, and brings administrative overhead if the number of the nodes and users in the network is large.
\item Even if the network staff would be able to both overcome this overhead and reconfigure the QoS mechanism on each of the network change, it would still be impossible to properly handle (and prevent) some important kinds of network disturbances: Typically, WiFi-based (and generally wireless) networks of some ISPs may change very frequently due to short (0.1s--1s) network disturbances caused by e.g. radio jams and atmospheric phenomena, which already have impact on latency-critical services. Centralized traffic scheduling is, by design, unable to properly react in such situations: The total time required to detect such disturbances reliably from a distance \emph{and} to reconfigure the centralized QoS mechanism to act accordingly is typically much higher than the actual duration of the disturbance.
\end{enumerate}

In this thesis, we describe, implement and benchmark a traffic scheduler that aims to improve this situation: It is simple enough to be placed at the exact bottleneck of the network where it can quickly and precisely react to network problems and prevent impact on the quality of delivered service; at the same time it supports a multi-flow multi-priority stochastical traffic scheduling that guarantees a level of fairness and service differentiation that is required for a typical ISP. The design is built on previous research in the area --- it combines ideas of Controlled Delay (CoDel) active queueing mechanism with Stochastic Fair Queueing (SFQ) and uses Deficit Round-robin (DRR) algorithm to provide fairness and configurable prioritization of more `important' traffic classes. \todo{tady to chce citace na vsechny ty ostatni schedulery}

The resulting traffic-scheduling algorithm is called Multilevel Stochastically Fair CoDel (MSFC). This thesis implements the algorithm in the environment of ns-3 network simulator \todo{cite} for benchmarking. An experimental implementation of MSFC for Linux operating system kernel network traffic-control infrastructure has previously been implemented by the supervisor of this thesis, but has not been rigorously tested nor seriously benchmarked. Providing the benchmarks and thus showing that the scheduler is suitable for real-world deployment is the main result of this thesis.


\section*{Related Work}
Researchers have realised the potential \todo{potential for what?} of traffic scheduling a long time ago and many traffic schedulers have emerged addressing different issues \todo{mozna lepsi zacit od toho ze existujou problemy, nez ze si researcheri neco uvedomili a pak najednou zacli resit problem}. The bufferbloat was first addressed by Random Early Detection (RED) \cite{Floyd:1993:RED:169931.169935}, that used average queue occupancy as marker of full queue. Its later improvements include Advanced RED \cite{Floyd01adaptivered:}, Robust RED \cite{RRED}, and CHoKe \cite{pan2000choke}. In 2012, Nichols, Kathleen and Van Jacobson proposed Controlled Delay (CoDel) algorithm, that uses time packets spend in queue as indication of bufferbloat.

Another issue \todo{`that the' (bez toho je to dvojznacny)} traffic schedulers address is fair approach to all users of the network \todo{to neni issue kdyz ten approach je fair :D mozna chces rict `fairness of the ...'?}. The concept of using one \todo{a separate one?} queue per \todo{each?} flow \todo{of what?} is called Fair Queueing (FQ) \XX{and Nagle first proposed in} \cite{Nagle:FQ} \todo{neni dobry pouzivat citace jako podstatny jmena, tohle se pise: FQ was first proposed by Nagle [5]}. Stochastic FQ (SFQ) \cite{SFQ} uses \todo{a} hash \todo{function! hash je lehka droga! :D} to distin\XX{ct} \todo{distinguish?} flows from each other. Bit-by-bit round-robin \cite{demers1989analysis} ensure\XX{d} \todo{can be used to ensure?} absolute fairness for all flows, Deficit round-robin (DRR) \cite{EffDRR} approximates it to achieve O(1) algorithm. FQ CoDel \cite{fq_codel} combines CoDel \XX{and DRR} \todo{with fair queueing AND drr?}.

The last \todo{A different?} category of traffic schedulers allows \XX{user} \todo{the administrator} to precisely  define the shape of the traffic. \todo{tadyto je trochu slozitejsi --- ty nahore jsou normalne oznaceny jako classless --- nepocitaj s tim ze traffic ma vic ruznejch druhu. Navic existuje i TBF kterym jde udelat i classless shaping. Rozdil CBQ, HTB a HFSC je ten, ze pocitaj ze uprostred jednoho scheduleru je vic ruznejch trid trafficu. Problem samozrejme je ze to nekdo musi (pomerne dost slozite) nastavit a ze to je vetsinou vypocetne narocnejsi nez ty classless.} This includes and Class Based Queueing (CBQ) \cite{CBQ}, Hierarchical Token Bucket (HTB) \cite{HTB},  Hierarchical Fair Service Curve (HFSC) \cite{HFSC}.

\section*{Layout of this Thesis}
This thesis is structered as follows: \todo{the} First chapter offers \todo{provides} a detailed overview of \XX{previous research} \todo{to je dost silny slovo, dal bych jen `popular approaches to...'} in traffic scheduling. It presents issues of the area as well as principles and algorithms that address them. \todo{the} Second chapter describes the design of the MSFC scheduler \XX{we study in this thesis} \todo{kdyz tomu das rovnou jmeno, tak pak nemusis psat ze to je furt `tenhle' :D} and describes its Linux and ns-3 implementations. The third chapter is dedicated \todo{dedication je ze to necemu vylozene venujes jako darek, mozna chces rict proste `the chapter concerns the simulations'} to simulations we performed in order to evaluate the scheduler performance and compare it to other traffic schedulers \todo{a rovnou rekni ze tam fakt sou i ty vysledky, protoze na ne kazdej ceka :D}.

