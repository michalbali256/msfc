Diferenciácia služieb, teda schopnosť mechanizmov zabezpečovania kvality služby (QoS) spĺňať rôzne požiadavky rôznych typov prenosu po sieti, je dôležitou súčasťou poskytovania internetových služieb. Bežné metódy zlepšovania kvality diferencovaných služieb vyžadujú centralizované plánovače sieťovej prevádzky, ktoré v sieťach typických ISP nemôžu reagovať na poruchy.

V tejto práci opisujeme, implementujeme a meriame výkon plánovača sieťovej prevádzky, ktorý je dostatočne jednoduchý na to, aby bol umiestnený v kritických miestach siete, kde môže presne reagovať na vzniknuté problémy v sieti; súčasne podporuje viacúrovňové stochastické plánovanie sieťovej prevádzky, ktoré zaručuje istú úroveň spravodlivosti v sieti a diferenciácie služieb. Návrh je inšpirovaný predchádzajúcim výskumom v oblasti --- kombinuje idey CoDelu a SFQ.

Výsledný návrh plánovača sieťovej prevádzky, nazvaný Multilevel Stochastically Fair CoDel (MSFC), implementujeme v sieťovom simulátore ns-3. Simulácie na sieti podobnej infraštruktúre ISP vykazujú v porovnaní s inými necentralizovanými plánovačmi zlepšenie kvality diferencovaných služieb.